\documentclass{article}

\title{Objective-C Language Proposal for non-Runtime Protocols}
\author{Nathan Lanza}
\date{April 9, 2020}

\begin{document}

\maketitle
\section{Motivation}

Objective-C allows the developer to query information about protocols at
runtime. The compiler emits various types of metadata for protocol declarations
including:

\begin{itemize}
  \item A \verb|cstring| for the protocol's name
  \item An \verb|_objc_method_list| of \verb|_objc_method|s for each method
    defined as part of the protocol. This creates four lists: optional-instance,
    non-optional-instance, optional-class and non-optional-class.
  \item Similar to the above bullet point for properties.
  \item A \verb|_protocol_list_t| for the protocols which this protocol inherits
    from.
  \item A list of the types of the methods that this protocol requires in
    Objective-C @encoding format.
  \item A \verb|_protocol_t| struct containing various other pieces of
    information along with pointers to all of the above info.
  \item An element in the \verb|__objc_protolist| section pointing to the
    \verb|_protocol_t|.
\end{itemize}

Ultimately, given extensive usage of protocols a program can end up with
megabytes worth of unused metadata needlessly bloating binary size.

We are proposing an annotation that can be applied to protocol declarations that
will cause the compiler to not emit any metadata. The protocol would still be
used by the frontend at compile time to perform all the usually typechecking,
but none of the above mentioned metadata would be emitted.

\section{Implementation}

The following attribute will be allowed on Objective-C protocol declarations to
declare them as non-runtime protocols:

\[ \verb|__attribute__((objc_non_runtime_protocol))| \]

This tells the compiler that it does not need to emit metadata associated with
the protocol. For example, when emitting the \verb|_class_ro_t| for a class we
must emit a \verb|_protocol_list_t| containing references to the
\verb|_protocol_t| for all the declared protocol conformances. When we walk
through this list we will check all the protocols for whether or not they are
attributed with this annotation and, if so, simply skip over this protocol.

There are various runtime APIs that will attempt to inspect the runtime metadata
that we are proposing to eliminate. There is also the compile time \linebreak
\verb|@protocol(SomeProtocol)| expression. This is trivially addressable as it's
codegen actually access the named protocol. Thus we can catch this in the
frontend. We currently emit an error diagnostic of ``can not use a protocol
declared \verb|objc_non_runtime_protocol| in an \verb|@protocol| expression''.

The Objective-C runtime also offers an \verb|objc_getProtocol| function that
accepts a \verb|char *| string and will lookup the protocol from the
\verb|__objc_protolist| section. For any given input string, if the
runtime can't find the protocol it will simply return \verb|null|. For a protocol
annotated with \verb|objc_non_runtime_protocol| this function would simply just
return \verb|null| as well since we would not be emitting anything to that
section.

There are many other APIs defined by the runtime that operate on protocols.
However, they all require a \verb|Protocol| handle as returned by one of the two
above mentioned APIs. So they would all not even be callable given no handle to
the protocol.

\subsection{Considerations for Swift}

Swift uses protocol metadata at runtime for various purposes. At first, the plan
for Swift interoptability would simple be to reject attempts to use ObjC
non-runtime protocols from Swift. Future considerations could be main for
properly using the protocol within Swift for compile-time typechecking in a
similar fashion to what's being proposed for Objective-C.

\section{Tools to Aid in Correct Annotation}

\subsection{Runtime Check Builds}

A build configuration can be defined to interpose the \verb|objc_getProtocol|
function to test the argument against a list of non-runtime attriubted protocols.
We can emit the list of attributed protocols into a seperate section that is
loaded at runtime. When the program attempts to call \verb|objc_getProtocol|
with the name of a protocol that was marked as non-runtime the program could
crash with an appropriate error.

\subsection{Whole-app Analysis}

This protocol metadata removal optimization is done in the frontend. But
information about the requested consumers of the runtime metadata of a protocol
is not globally known in the frontend.

We can define a whole-app analysis LTO pass which tracks all protocols that had
runtime APIs used against them. If a given protocol's metadata is not observed
to be used during this analysis we can suggest to the developer that a protocol
could be annotated as a non-runtime candidate. This could possible be integrated
into various IDEs such as Xcode and Visual Studio Code.

\subsection{Statically Inferable Cases}

There are also cases where it is entirely statically inferable whether or not
the protocol would make sense to mark as non-runtime. If a protocol is defined
in a `\verb|.m| or \verb|.mm| file then (in most cases) this won't be visible
in any other translation unit. If it is not used dynamically then it would be
valid to mark it as non-runtime.

\end{document}
